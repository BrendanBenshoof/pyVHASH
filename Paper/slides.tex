\documentclass[8pt]{beamer}
\usetheme{Dresden}

\usepackage[latin5]{inputenc}
\usepackage[english]{babel}
\usepackage{amsmath}
\usepackage{amsfonts}
\usepackage{amssymb}
\usepackage{graphicx}
\usepackage{subcaption}
%\usepackage{subfig} 
\usepackage{algorithm} 
\usepackage{algorithmic}

\title{A Distributed Greedy Heuristic for Computing Voronoi Tessellations With Applications Towards Peer-to-Peer Networks}
\author{Brendan Benshoof \qquad Andrew Rosen \qquad \\Anu G. Bourgeois \qquad Robert W. Harrison \\Department of Computer Science, Georgia State University}
%\subtitle{}
%\logo{}
%\institute{}
%\date{}
%\subject{}
%\setbeamercovered{transparent}
%\setbeamertemplate{navigation symbols}{}

\begin{document}
	\maketitle
	
	
	
\begin{frame}{Outline}
	\tableofcontents
\end{frame}
	
	
\section{Background}
\subsection{Motivation}
	
	\begin{frame}{Motivation}
		We were creating a distributed network based off of Delaunay Triangulation and needed a fast distributed algorithm for calculating Delaunay peers.
		We ran into a couple of issues.
		\begin{itemize}
			\item Distributed algorithms for solving Delaunay triangulation don't really exist (every node does a global solution).
			\item Not any fast solutions if we start moving out of 2D Euclidean space.
		\end{itemize}
		This leaves approximation.  However:
		\begin{itemize}
			\item Simple approximation doesn't guarantee a fully-reachable network ($k$-nearest neighbor, nodes in radius $r$).
			\item Other solutions \cite{voronet} required prohibitively high sampling or other hidden time costs.
		\end{itemize}
		And nothing can handle moving nodes.
	\end{frame}		
	
	\begin{frame}{Voronoi Tessellation and Delaunay Triangulation}
		\begin{itemize}
			\item Voronoi Tessellation
				\begin{itemize}
					\item Partition space such that each point is mapped the 'voronoi generator' to which it is closest.
				\end{itemize}
			\item Delaunay Triangulation
				\begin{itemize}
					\item generate a triangulation, such that, for every triangle the circumcircle does not contain any other points 
				\end{itemize}
		\end{itemize}
	\end{frame}
			
			
	\begin{frame}{Voronoi Example}
	\begin{figure}
	\centering
	\includegraphics[width=0.5\linewidth]{new_voronoi}
	\caption{The space is partitioned, such that each point in a partition is closest to that partitions 'Voronoi generator'}
	\label{voronoi}
\end{figure}
	\end{frame}

	\begin{frame}{Delaunay Example}
	\begin{figure}
	\centering
	\includegraphics[width=0.5\linewidth]{delaunay}
	\caption{The space is partitioned, such that each point in a partition is closest to that partitions 'Voronoi generator'}
	\label{delaunay}
\end{figure}
	\end{frame}
				
		
		
	
\subsection{Distributed Hash Tables}
	\begin{frame}{Distributed Hash Tables}
		\begin{itemize}
			\item Abstractly, a DHT is a mechanism for maintaining a large state in a decentralized network.
			\item In practice, the state is a large number of $ (key, value) $ records.
			\item A Distributed hash table assigns those records to servers and routes request for those records to those servers.
			\item Current incarnations of Distributed hash tables assign servers and records locations in an arbitrary metric space.
			\item Servers are assigned responsibility for records that are ``close'' to them, and to peer with ``nearby'' servers.
			\item DHTs currently use a variety metric spaces.
		\end{itemize}


	\end{frame}

	\begin{frame}{Applications of DHTs}
		\begin{itemize}
			\item \textit{P2P file sharing} is by far the most prominent use of DHTs.  
			The most well-known application is BitTorrent \cite{bittorrent}.
			\item \textit{Distributed Domain Name Systems} (DNS) have been built upon DHTs \cite{cox2002serving} \cite{pappas2006comparative}.
			Distributed DNSs are much more robust that DNS to orchestrated attacks, but otherwise require more overhead.
			%\item Distributed search (faroo)
			\item Distributed \textit{machine learning} \cite{liparameter}.
			\item Many \textit{botnets} are now P2P based and built using well established DHTs \cite{saad2011detecting}. 
			This is because the decentralized nature of P2P systems means there's no single vulnerable location in the botnet.
		\end{itemize}
	\end{frame}

\begin{frame}{Extant Varieties of DHT}
	\begin{itemize}
		\item Ring Based DHTs
		\begin{itemize}
			\item Chord
			\item Pastry
			\item Tapestry
		\end{itemize}
		\item Tree Based DHTs
		\begin{itemize}
			\item CAN
			\item Kademlia
		\end{itemize}
	\end{itemize}
\end{frame}

\begin{frame}{How are DHTs and Vonroi Tesselation/Delunay Trianguation related?}
	A Server is responsible for records ''close`` to it (Voronoi Triangulation)
	A Server peers with other servers that bound it's Voronoi Region (Delunay Triangulation)
	DHTs often have peers in excess of the Delaunay triangulation to shorten lookups, however the peers that are the Delunay neighbors
	\begin{itemize}
		\item Ring Based DHTs
		\begin{itemize}
			\item Chord: a unidirectional modulus ring metric
			\item Pastry, Symphony: bidirectional modulus ring
		\end{itemize}
		\item Tree Based DHTs
		\begin{itemize}
			\item CAN: Euclidean distance
			\item Kademlia: XOR distance
		\end{itemize}
	\end{itemize}
\end{frame}


\begin{frame}{Why do we need a distributed Voronoi heuristic?}
	\begin{itemize}
		\item The different topologies DHTs utilize present optimization trade-offs (lookup latency, number of lookup hops, network robustness, availability, processing overhead)
		\item The primary effort in implementing a new metric space in a DHT is implementing Voronoi Regions/Delunay Triangulation algorithm in that metric.
		\item DGVH allows many metrics to be tested without requiring the design and development effort of generating an exact Voronoi Regions/Delunay Triangulation algorithm.
	\end{itemize}
	
	So we created a Distributed Greedy Voronoi Heuristic, or DGVH for short.
\end{frame}



	

	
\section{DGVH}
	\begin{frame}{Distributed Greedy Voronoi Heuristic}
		\begin{itemize}
			\item Geometrically intuitive method of approximating the one-hop delaunay peers of a Node
			\item Is guaranteed to form a connected mesh (unlike k-nearest heuristic)
			\item Can be utilized in any continuous metric space
		\end{itemize}
		
	\end{frame}
	
	\subsection{Our Heuristic}

	\begin{frame}{DGVH Intuition}

		\begin{itemize}
			\item The majority of Delunay links cross the corresponding Voronoi edges.
			\item We can test if the midpoint between two potentially connecting nodes is on the edge of the voronoi region.
			\item This intuition fails if the midpoint between two nodes does not fall on their voronoi edge.
		\end{itemize}

	\end{frame}
	
	
	\begin{frame}{DGVH Algorithim}

			\begin{algorithmic}[1]  % the numberis how many lines
				\STATE Given node $n$ and its list of $candidates$.
				\STATE $peers \leftarrow$ empty set that will contain $n$'s one-hop peers
				\STATE Sort $candidates$ in ascending order by each node's distance to $n$
				\STATE Remove the first member of $candidates$ and add it to $peers$
				\FORALL{$c$ in $candidates$}
				\STATE $m$ is the midpoint between $n$ and $c$
				\IF{Any node in $peers$ is closer to $m$ than $n$}
				\STATE Reject $c$ as a peer
				\ELSE
				\STATE Remove $c$ from $candidates$
				\STATE Add $c$ to $peers$
				\ENDIF
				\ENDFOR
			\end{algorithmic}

	\end{frame}
	
	
	
\begin{frame}{DGVH Example}
	\begin{figure}
	\centering
	\includegraphics[width=0.5\linewidth]{DGVH}
	\caption{'DGVH connects all nodes where the midpoint falls on the border of the voronoi regions'}
	\label{delaunay}
\end{figure}
	\end{frame}
	
	
	
	
\subsection{Peer Management}
	\begin{frame}{Realistic Candidate set Size}
		\begin{itemize}
			\item In application, a single node will not need to calculate the triangulation for every peer in the network, rather it will only calculate it's own peers.
			\item A smart ''join`` process will prevent the Candidate Set from reaching $O(n)$
			\item Expected size of the candidate set is $O(degree^2)$ which compromises current peers and 2-hop peers
			\item The average degree of many metric spaces is $O(1)$, then therefore DGVH will practically run in $O(1)$ time in those metric spaces.
		\end{itemize}
	\end{frame}
	\begin{frame}{Error Mitigation}
		\begin{itemize}
			\item As DGVH is a heuristic, it will likely have errors when compared to a global Delaunay triangulation.
			\item This error rate is difficult to find analytically and will vary between metric spaces and distributions of locations
			\item A simple method of mitigating this error is to keep both one and two hop peers.
		\end{itemize}
	\end{frame}
	
\subsection{Algorithm Analysis}
	
	
\section{Experiments}
	
\subsection{Heuristic Accuracy}

	\begin{frame}{Experiment 1}
		In our first experiment, we wanted to test the accuracy of our heuristic.
		\begin{itemize}
			\item We generated a random graph and created the Delaunay triangulation using a global solution and DGVH.
			\item We found DGVH had approximately one error per node.
		\end{itemize}
		
		Our second set of experiments demonstrate that this error rate is sufficiently low for distributed applications.
	\end{frame}


\begin{frame}{Results}
\begin{figure}
	\centering
	\includegraphics[width=0.5\linewidth]{error_rate}
	\caption{As the size of the graph increases, we see approximately 1 error per node.
	We can also see that the error rate and number of nodes has a linear relationship.}
	\label{exp_0}
\end{figure}
\end{frame}

\subsection{Routing Accuracy}


\begin{frame}{Experiment 2}
Our second experiment were designed to test how well nodes could form a routing topology using DGVH.

\begin{itemize}
	\item For each trial, we create a random graph.
	\item During the first two cycles, nodes are given 10 random connections.
	\item In each cycle (including the first two):
	\begin{itemize}
		\item Each node gossips and runs DGVH.
		\item 2000 random lookups from random nodes to random locations performed.
	\end{itemize}
	\item The rate of successful lookups approaches 1.0 as time progresses.
\end{itemize}
\end{frame}



\begin{frame}{Results}
\begin{figure}
 	\label{fig:conv}
 		\begin{subfigure}[b]{.45\linewidth}
 			\centering
 			\includegraphics[height=3cm]{conv_d2}
 			%\caption{This plot shows the accuracy rate of lookups on a 2-dimensional network as it self-organizes.}
 			\label{conv2}
 		\end{subfigure} 
 		\begin{subfigure}[b]{.45\linewidth}
 			\centering
 			\includegraphics[height=3cm]{conv_d3}
 			%\caption{This plot shows the accuracy rate of lookups on a 3-dimensional network as it self-organizes.}
 			\label{conv3}
 		\end{subfigure}\\
 		\begin{subfigure}[b]{.45\linewidth}
 		\centering
 			\includegraphics[height=3cm]{conv_d4}
 			%\caption{This plot shows the accuracy rate of lookups on a 4-dimensional network as it self-organizes.}
 			\label{conv4}
 		\end{subfigure}
 		\begin{subfigure}[b]{.45\linewidth}
 			\centering
 			\includegraphics[height=3cm]{conv_d5}
 			%\caption{This plot shows the accuracy rate of lookups on a 5-dimensional network as it self-organizes.}
 			\label{conv5}
 		\end{subfigure}
 		
 	\caption{These figures show, starting from a randomized network, DGVH forms a stable and consistent network topology.
 		The Y axis is the percentage of successful lookups out of 2000 queries and the X axis is the number of gossips cycles.}
 	
 \end{figure}
	


\end{frame}	

\section{Conclusion}
	\begin{frame}{Other Applications}
		One type of distributed system that can use Voronoi tessellations are \textit{wireless ad-hoc networks}.
		\begin{itemize}
			\item Solves the coverage-boundary problem \cite{carbunar2004distributed}.
			\item Can be used for sleep scheduling \cite{chen2008voronoi}
		\end{itemize}
	\end{frame}	
	
	\begin{frame}{Conclusions}
		\begin{itemize}
			\item DGVH provides a simple approximation Voronoi Triangulation / Delaunay Triangulation, of similar complexity to picking k-nearest node or nodes in distance k.
			\item Unlike other simple approximation methods, DGVH guarantees a fully connected graph.
			\item In practice, DGVH supplemented with K-nearest peers approximation provides an accurate, fast and distributable computation, with a fully connected graph as result.
		\end{itemize}
	\end{frame}		

\bibliography{P3DNS}
\bibliographystyle{plain}
	
\end{document}