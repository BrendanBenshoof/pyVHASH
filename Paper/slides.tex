\documentclass[8pt]{beamer}
\usetheme{Dresden}

\usepackage[latin5]{inputenc}
\usepackage[english]{babel}
\usepackage{amsmath}
\usepackage{amsfonts}
\usepackage{amssymb}
\usepackage{graphicx}

\title{A Distributed Greedy Heuristic for Computing Voronoi Tessellations With Applications Towards Peer-to-Peer Networks}
\author{Brendan Benshoof \qquad Andrew Rosen \qquad \\Anu G. Bourgeois \qquad Robert W. Harrison \\Department of Computer Science, Georgia State University}
%\subtitle{}
%\logo{}
%\institute{}
%\date{}
%\subject{}
%\setbeamercovered{transparent}
%\setbeamertemplate{navigation symbols}{}

\begin{document}
	\maketitle
	
	
	
	\begin{frame}{Outline}
		\tableofcontents
	\end{frame}
	
	
	\section{Background}
	
	\begin{frame}{Distributed Hash Tables}
		\begin{itemize}
			\item Abstractly, a DHT is a mechanism for maintaining a large state in a decentralized network.
			\item In practice, the state is a large number of key,value records.
			\item A Distributed hash table assigns those records to servers and routes request for those records to those servers.
			\item Current incarnations of Distributed hash tables assign servers and records locations in an arbitrary metric space.
			\item Servers are assigned responsibility for records that are ``close'' to them, and to peer with ``nearby'' servers.
			\item DHTs currently use a variety metric spaces.
		\end{itemize}


	\end{frame}

	\begin{frame}{Applications}
		\begin{itemize}
			\item \textit{P2P file sharing} is by far the most prominent use of DHTs.  
			The most well-known application is BitTorrent \cite{bittorrent}.
			\item \textit{Distributed Domain Name Systems} (DNS) have been built upon DHTs \cite{cox2002serving} \cite{pappas2006comparative}.
			Distributed DNSs are much more robust that DNS to orchestrated attacks, but otherwise require more overhead.
			%\item Distributed search (faroo)
			\item Distributed \textit{machine learning} \cite{liparameter}.
			\item Many \textit{botnets} are now P2P based and built using well established DHTs \cite{saad2011detecting}. 
			This is because the decentralized nature of P2P systems means there's no single vulnerable location in the botnet.
		\end{itemize}
	\end{frame}

\begin{frame}{Extant Varieties of DHT}
	\begin{itemize}
		\item Ring Based DHTs
		\begin{itemize}
			\item Chord
			\item Pastry
			\item Tapestry
		\end{itemize}
		\item Tree Based DHTs
		\begin{itemize}
			\item CAN
			\item Kademlia
		\end{itemize}
	\end{itemize}
\end{frame}

\begin{frame}{How are DHTs and Vonroi Tesselation/Delunay Trianguation related?}
	A Server is responsible for records ''close`` to it (Voronoi Triangulation)
	A Server peers with other servers that bound it's Voronoi Region (Delunay Triangulation)
	DHTs often have peers in excess of the Delaunay triangulation to shorten lookups, however the peers that are the Delunay neighbors
	\begin{itemize}
		\item Ring Based DHT
		\begin{itemize}
			\item Provide a unidirectional modulus ring 
		\end{itemize}
		\item Tree Based DHTs
		\begin{itemize}
			\item CAN
			\item Kademlia
		\end{itemize}
	\end{itemize}
\end{frame}

	\subsection{Motivation}
\begin{frame}{Why do we need DGVH?}
	The different topologies DHTs utilize present optimization tradeoffs (lookup latency, number of lookup hops, network robustness, availability, processing overhead)
	The primary effort in implementing a new metric space in a DHT is implementing Voronoi Regions/Delunay Triangulation algorithm in that metric
\end{frame}

	
	
	\subsection{Related Work}
	
	\section{DGVH}
	
	\subsection{Our Heuristic}
	
	\subsection{Peer Management}
	
	
	\subsection{Algorithm Analysis}
	
	
	\section{Experiments}
	
	
	
	\section{Conclusion}
	
	
	
	\begin{frame}
		\frametitle{}
	\end{frame}
	
\end{document}