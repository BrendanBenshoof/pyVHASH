\documentclass{IEEEtran}

\usepackage[utf8]{inputenc} % set input encoding (not needed with XeLaTeX)
%\usepackage{geometry}
%\geometry{letterpaper}
%%% Examples of Article customizations
% These packages are optional, depending whether you want the features they provide.
% See the LaTeX Companion or other references for full information.



\usepackage{graphicx} % support the \includegraphics command and options

% \usepackage[parfill]{parskip} % Activate to begin paragraphs with an empty line rather than an indent

%%% PACKAGES
\usepackage{url}
\usepackage{booktabs} % for much better looking tables
\usepackage{array} % for better arrays (eg matrices) in maths
\usepackage{paralist} % very flexible & customisable lists (eg. enumerate/itemize, etc.)
\usepackage{verbatim} % adds environment for commenting out blocks of text & for better verbatim
\usepackage{subfig} % make it possible to include more than one captioned figure/table in a single float
\usepackage{algorithm} 
\usepackage{algorithmic}
\usepackage{amsmath}



% These packages are all incorporated in the memoir class to one degree or another...

%%% HEADERS & FOOTERS
%\usepackage{fancyhdr} % This should be set AFTER setting up the page geometry
%\pagestyle{fancy} % options: empty , plain , fancy
%\renewcommand{\headrulewidth}{0pt} % customise the layout...
%\lhead{}\chead{}\rhead{}
%\lfoot{}\cfoot{\thepage}\rfoot{}

%%% END Article customizations

%%% The "real" document content comes below...

\title{VHash: An Optimized Voronoi-Based Distributed Hash Table }
\author{Double Blind}
%\author{Brendan Benshoof \qquad Andrew Rosen  \\Department of Computer Science, Georgia State University\\  bbenshoof@cs.gsu.edu \qquad rosen@cs.gsu.edu }
\date{} % Activate to display a given date or no date (if empty),
         % otherwise the current date is printed 

\begin{document}
\maketitle

\begin{abstract}
%1. State the problem
Distributed Hash Tables (DHT) provide a fast and robust decentralized means of key-value storage and retrieval and are typically used in Peer-to-Peer applications.
DHTs assign nodes a single identifier derived from the hash of their IP address and port, which results in a random overlay network.   
%2. Say why it’s an interesting problem
A random overlay network is not explicitly optimized for certain metrics, such as latency, trust, energy, or hops on an overlay network.  It is often desirable to generate an overlay network that is optimized for one or more specific metrics.

%3. Say what your solution achieves
This paper presents VHash, a DHT protocol to construct an overlay optimized for such metrics. 
VHash exploits a fast and efficient Delaunay Triangulation heuristic in a geometric space. 
%4. Say what follows from your solution
We used VHash to generate an overlay with edges that minimize latency. 
While we focused on latency in this paper, VHash optimizes on any defined metrics.
This overlay outperformed an overlay generated by the Chord DHT protocol in terms of lookup time.
VHash provides a robust, scalable, and efficient distributed lookup service.

\end{abstract}
%  We could use a different config 
% networks are only hpped
% we can do non-euclidean metrics if we have a non-euclidean distance and midpoint definition

\section{Introduction}
A Distributed Hash Table is used provide an overlay network for many P2P applications. %State of the art DHT techniques are built on trees or ring structures to ensure that the routing distance is $O(lg(n))$ hops between nodes. 
%keep?
%In the vast majority of Distributed Hash Tables, such as Chord \cite{chord}, Kademlia\cite{kademlia}, Pastry \cite{pastry}, a node is mapped key on a 1-dimensional keyspace.  This key is chosen via a hash function, such as SHA-1, ensuring that nodes are randomly and uniformly the overlay network.
%keep?
%This provides the network with load balancing properties.  Since files are assigned keys in a similar manner to nodes, the node are distributed evenly ac.  The distribution also provides fault tolerance;  if all the nodes located in a real geographic region were suddenly taken offline, the damage to the network would be spread uniformly throughout the network and maintenance would repair the damage.%
Each node in the overlay network maintains a routing table of a subset of other nodes in the overlay.  
The configuration and rules of the routing table vary from one protocol and another;  the entries of the table might be serperated by powers of 2 \cite{Chord}, be determined by shared prefixes \cite{pastry}, or be chosen according to a probablistic distribution \cite{kleinberg2000navigation}, but the goal is to  minimize the number of overlay hops the distributed lookup needs to make.


`distant 

Was there some way to encode latency data as part of the overlay's topology? We can't do that without jeopardizing the load balance and fault tolerance properties if we try to modify or replace the hash function.  If we could generate more than one coordinate for each object in the onverlay, with each coordiate corresponding to some key or network metric, then we can used a distance functuon to optimize over that overlay.

We don't assign latency we embded it.
%What if we added additional axises to the traditional  DHT 

%These topologies, while sufficient in reasonably local networks, do not take embed the lengths or latencies of routes defined by the topology and assume that every hop has similar latency and throughput. For a global network, a more intelligent means of generating a dynamic overlay network with efficient routing, storage, and backups is needed for future P2P applications.


We present VHash as a DHT designed to take inter-node latency information into account when generating an overlay on a massive scale.  VHash creates an approximation of a Voronoi network to define the routing tables and dictate where content is stored in the network.  We accomplish this by assigning each node $d$ coordinates, rather than than a single key.  The naive method of doing so is to assign coordinates to servers based on the geographic location of nodes. More complex approaches approximate a minimum latency space based on internode latency.

Abstraction of the DHT concept.

Our paper presents the following:
\begin{itemize}
	\item We describe the VHash protocol and the underlying approximation algorithm for quick and efficient Voronoi region approximations.   Our approximation is simple, distributed, and greedy and accurately approximates the Voronoi region of objects in a space with an arbitrary number of dimensions, something than would require a great deal of computation.  This provides us with an overlay for embedding network metrics.
	\item We created a simulation to prove that the overlays created by VHash are accurate enough for routing messages from arbitrary source nodes to random destination locations.  We also show that by embedding the underlay 
	\item We compare the VHash protocol to the other protocols that are based off of Voronoi region approximation.  We also contrast the properties of VHash with well known extant protocols.
	\item We present the related work and discuss what plans we  have for embedding different problems in VHash.
\end{itemize}


\section{VHash}



\section{Simulations}
Scale free networks are internet shaped\cite{cohen2000resilience}



% Routing convergence tests

% Hop distance test

\section{Related Work}

\section{Future Work}

\bibliographystyle{ieeetr}
\bibliography{P3DNS}
\end{document}
