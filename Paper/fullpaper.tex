\documentclass{IEEEtran}

\usepackage[utf8]{inputenc} % set input encoding (not needed with XeLaTeX)
%\usepackage{geometry}
%\geometry{letterpaper}
%%% Examples of Article customizations
% These packages are optional, depending whether you want the features they provide.
% See the LaTeX Companion or other references for full information.



\usepackage{graphicx} % support the \includegraphics command and options

% \usepackage[parfill]{parskip} % Activate to begin paragraphs with an empty line rather than an indent

%%% PACKAGES
\usepackage{url}
\usepackage{booktabs} % for much better looking tables
\usepackage{array} % for better arrays (eg matrices) in maths
\usepackage{paralist} % very flexible & customisable lists (eg. enumerate/itemize, etc.)
\usepackage{verbatim} % adds environment for commenting out blocks of text & for better verbatim
\usepackage{subfig} % make it possible to include more than one captioned figure/table in a single float
\usepackage{algorithm} 
\usepackage{algorithmic}
\usepackage{amsmath}



% These packages are all incorporated in the memoir class to one degree or another...

%%% HEADERS & FOOTERS
%\usepackage{fancyhdr} % This should be set AFTER setting up the page geometry
%\pagestyle{fancy} % options: empty , plain , fancy
%\renewcommand{\headrulewidth}{0pt} % customise the layout...
%\lhead{}\chead{}\rhead{}
%\lfoot{}\cfoot{\thepage}\rfoot{}

%%% END Article customizations

%%% The "real" document content comes below...

\title{VHash: An Optimized Voronoi-Based Distributed Hash Table }
\author{Double Blind}
%\author{Brendan Benshoof \qquad Andrew Rosen  \\Department of Computer Science, Georgia State University\\  bbenshoof@cs.gsu.edu \qquad rosen@cs.gsu.edu }
\date{} % Activate to display a given date or no date (if empty),
         % otherwise the current date is printed 

\begin{document}
\maketitle

\begin{abstract}
%1. State the problem
Distributed Hash Tables (DHT) provide a fast and robust decentralized means of key-value storage and retrieval and are typically used in Peer-to-Peer applications.
DHTs assign nodes a single identifier derived from the hash of their IP address and port, which results in a random overlay network.   
%2. Say why it’s an interesting problem
A random overlay network is not explicitly optimized for certain metrics, such as latency, trust, energy, or hops on an overlay network.  It is often desirable to generate an overlay network that is optimized for one or more specific metrics.

%3. Say what your solution achieves
This paper presents VHash, a DHT protocol to construct an overlay optimized for such metrics. 
VHash exploits a fast and efficient Delaunay Triangulation heuristic in a geometric space. 
%4. Say what follows from your solution
We used VHash to generate an overlay with edges that minimize latency. 
While we focused on latency in this paper, VHash optimizes on any defined metrics.
This overlay outperformed an overlay generated by the Chord DHT protocol in terms of lookup time.
VHash provides a robust, scalable, and efficient distributed lookup service.

\end{abstract}
%  We could use a different config 
% networks are only hpped
% we can do non-euclidean metrics if we have a non-euclidean distance and midpoint definition
\section{Introduction}

\cite{chord} \cite{kademlia} \cite{pastry}

Our presents the following contributions
\begin{itemize}
	\item VHash Protocol	
	\item Simulations
	\item Stuff we plan on doing with our awesome new toy
\end{itemize}


\section{VHash}



\section{Simulations}


% Routing convergence tests

% Hop distance test

\section{Related Work}

\section{Future Work}

\bibliographystyle{ieeetr}
\bibliography{P3DNS}
\end{document}
