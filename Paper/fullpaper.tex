\documentclass{IEEEtran}

\usepackage[utf8]{inputenc} % set input encoding (not needed with XeLaTeX)
%\usepackage{geometry}
%\geometry{letterpaper}
%%% Examples of Article customizations
% These packages are optional, depending whether you want the features they provide.
% See the LaTeX Companion or other references for full information.



\usepackage{graphicx} % support the \includegraphics command and options

% \usepackage[parfill]{parskip} % Activate to begin paragraphs with an empty line rather than an indent

%%% PACKAGES
\usepackage{url}
\usepackage{booktabs} % for much better looking tables
\usepackage{array} % for better arrays (eg matrices) in maths
\usepackage{paralist} % very flexible & customisable lists (eg. enumerate/itemize, etc.)
\usepackage{verbatim} % adds environment for commenting out blocks of text & for better verbatim
\usepackage{subfig} % make it possible to include more than one captioned figure/table in a single float
\usepackage{algorithm} 
\usepackage{algorithmic}
\usepackage{amsmath}



% These packages are all incorporated in the memoir class to one degree or another...

%%% HEADERS & FOOTERS
%\usepackage{fancyhdr} % This should be set AFTER setting up the page geometry
%\pagestyle{fancy} % options: empty , plain , fancy
%\renewcommand{\headrulewidth}{0pt} % customise the layout...
%\lhead{}\chead{}\rhead{}
%\lfoot{}\cfoot{\thepage}\rfoot{}

%%% END Article customizations

%%% The "real" document content comes below...

\title{VHash: An Optimized Voronoi-Based Distributed Hash Table }
\author{Double Blind}
%\author{Brendan Benshoof \qquad Andrew Rosen  \\Department of Computer Science, Georgia State University\\  bbenshoof@cs.gsu.edu \qquad rosen@cs.gsu.edu }
\date{} % Activate to display a given date or no date (if empty),
         % otherwise the current date is printed 

\begin{document}
\maketitle

\begin{abstract}
%1. State the problem
Distributed Hash Tables (DHT) provide a fast and robust decentralized means of key-value storage and retrieval and are typically used in Peer-to-Peer applications.
DHTs assign nodes a single identifier derived from the hash of their IP address and port, which results in a random overlay network.   
%2. Say why it’s an interesting problem
A random overlay network is not explicitly optimized for certain metrics, such as latency, trust, energy, or hops on an overlay network.  It is often desirable to generate an overlay network that is optimized for one or more specific metrics.

%3. Say what your solution achieves
This paper presents VHash, a DHT protocol to construct an overlay optimized for such metrics. 
VHash exploits a fast and efficient Delaunay Triangulation heuristic in a geometric space. 
%4. Say what follows from your solution
We used VHash to generate an overlay with edges that minimize latency. 
While we focused on latency in this paper, VHash optimizes on any defined metrics.
This overlay outperformed an overlay generated by the Chord DHT protocol in terms of lookup time.
VHash provides a robust, scalable, and efficient distributed lookup service.

\end{abstract}
%  We could use a different config 
% networks are only hpped
% we can do non-euclidean metrics if we have a non-euclidean distance and midpoint definition

\section{Introduction}
A Distributed Hash Table is used provide an overlay network for many P2P applications. %State of the art DHT techniques are built on trees or ring structures to ensure that the routing distance is $O(lg(n))$ hops between nodes. 
%keep?
%In the vast majority of Distributed Hash Tables, such as Chord \cite{chord}, Kademlia\cite{kademlia}, Pastry \cite{pastry}, a node is mapped key on a 1-dimensional keyspace.  This key is chosen via a hash function, such as SHA-1, ensuring that nodes are randomly and uniformly the overlay network.
%keep?
%This provides the network with load balancing properties.  Since files are assigned keys in a similar manner to nodes, the node are distributed evenly ac.  The distribution also provides fault tolerance;  if all the nodes located in a real geographic region were suddenly taken offline, the damage to the network would be spread uniformly throughout the network and maintenance would repair the damage.%
Each node in the overlay network maintains a routing table of a subset of other nodes in the overlay.  
The configuration and rules of the routing table vary from one protocol to another.
In a DHT the entries of the routing tables may be separated by powers of 2 \cite{chord}, be determined by shared prefixes \cite{pastry}, or be chosen according to a probabilistic distribution \cite{kleinberg2000navigation}, but the goal is to minimize the number of overlay hops the distributed lookup needs to make. 


%The ``problem''
%This is the correct choice make if there is no other information to act on during lookup or routing. 
A routing table created to minimize overlay hops does not necessarily create routes with minimized latency.
What if more information about the nodes could be encoded as part of the overlay, such as the node's latency?
Problems like latency require embedding a graph into cordinates in the metric space such as to minimize errors in inter-node distance.
If each node is defined as object in this space we can then use a distance function to choose the shortest path over these metrics and approximate a desired behavoir, in this case minimum latency.


This opens up two problems:
How to de create and maintain a DHT based on an arbitrary metric space?
How do we usefully embed inter-node latency and other important rounting information into that metric space's coordiante system?

VHash is a DHT designed to take inter-node latency information into account when generating an overlay.
Other network metrics could easily be applied to VHash.
VHash creates an approximation of a Voronoi network and Delaunay Triangulation to define the routing tables and dictate where content is stored in the network.
We accomplish this by assigning each node $d$ coordinates, rather than than a single key.
We use a basic spring model to embed inter-node latency graph in the overlay.
A result of this is that nodes in VHash  can \emph{move} through the metric space over the lifetime of the node;  their position is not nessarily fixed.
Our paper presents the following:

\begin{itemize}
	\item We describe the VHash protocol (Section II) and the underlying approximation algorithm of overlay's Delaunay Trinangulation and the corresponding Voronoi diagrams.
    Our approximation is distributed, greedy, efficient, and accurate in arbitrary number of dimensions and creates an overlay with edges that minimize distance over network metrics while maintaining robustness, scalablilty, and polylogarithmic lookup time in overlay hops.
	\item We use VHash to provide us with an overlay for embedding network metrics.
    We present our basic spring model for embedding nodes in the overlay with latency information and discuss other network metrics that can be used with VHash (Section III).
	\item We created simulations (Section IV) to prove that the overlays created by VHash are accurate enough for routing messages from arbitrary source nodes to random destination locations.
    We also show that by embedding the latency graph, our routing dramatically outperforms Chord (and by extension, other overlays with $\lg(n)$-sized routing tables) in terms of latency.
	\item We compare the VHash protocol to the other protocols that are based off of Voronoi region approximation.  We also contrast the properties of VHash with well known extant protocols.
	\item We discuss future work that follows from our what plans we have for embedding different problems using VHash.
\end{itemize}


\section{VHash}
Nodes in the VHash network periodically gossip with other nodes, exchanging information about theirs peers and use the approximation algorithm to refine its list of neighbors.
These neighbors approximate the node's Voronoi region and it's corresponding responsibilities.
This approximation algorithm is fast and can be used for spaces with an arbitrary number of dimensions.

\subsection{Voronoi Regions in DHTs}
A Voronoi diagram is the division of a space into cells or regions along a set of objects $O$ such that all the points in a particular region are closer to one object than any other object.  
We refer to the region owned by an object as that object's Voronoi region.
The Delaunay Trangulation of this same space along the same set of objects is defined by the edges such that no object is inside the circumcircle of any triangle formed by the edges\cite{geoalg}.  
The Voronoi diagram and Delaunay Triangulation are dual problems, as an edge between two objects in a Delaunay Triangulation exists iff those object's Voronoi region border each other.  
This means that solving either problem will yield the solution to both. 

In our network, the nodes are the objects of the Voronoi diagram and their regions define the regions they are responsible for.  The edges created by the Deluanay Triangulation correspond to the connections between neighboring nodes.  Computing Voronoi diagrams in a distributed and genralized fashion is prohbitively time expensive.
A greedy approximation of the Voronoi regions is sufficient for the VHash protocol.
We created a new greedy, online algorithm that approximates and maintains the set of peers defining the node's Voronoi region and Delaunay Triangulation.

A formal and thorough description of Voronoi diagrams as well as their applications can be found in \cite{aurenhammer1991voronoi}.

Arguably all DHTs are built on the concept of Vornoi regions.
In all DHT's currently contrived, a node is responsible for all points in it's hash space to which it is the ``closest'' node.
These DHTs have carefully chosen metric spaces such that these regions are very simple to calculate.
For example Chord and similar ring based DHTs utilize a unidirectional one dimensional ring as thier metric space, such that the region for which a node is responsible is the region between it and it's predecessor.
VHash generalizes these behavoirs.
By choosing a particular metric space VHash can approximate other DHTs.
The toroidal metric space utilized in this paper is an extension of the ring topology to additional dimensions. 

The cost of generalizing VHash to utilize any metric space is the efficency of calculating vornoi regions. It is prohbitively expensive to generate an exact calculation of the delunay peers and vornoi regions of a metric space in a distributed fashion\cite{raynet}%\footnote{They cite Geometric Algorithms by Boissonat}. 
Previous explorations into distributed approximations of node's voronoi regions (ray-net) offer constant time approximations.
We utilize a greedy heuristic that requires a similar computation cost of a single sample from ray-net's monte-carlo approximation.
As a heuristic, there exist edge cases where it is incorrect.
We show in the experimental section that this loss in accuracy has negligable impact on use.

\subsection{Distributed Greedy Voronoi Heuristic}
The Distributed Greedy Voronoi Heuristic (Algorithm \ref{DGVH}) is a fast method for nodes to select peers from thier Delunay Triangulation.
It is predicated on the assumption that the midpoint between two nodes falls on the line segment that they share on the edges of thier voronoi regions.
This holds true in the vast majority of cases.


\begin{algorithm}
\caption{Distributed Greedy Voronoi Heuristic}
\label{DGVH}
\begin{algorithmic}[1]  % the numberis how many lines
	\STATE $candiates$ is the set of candidate peers obtained from gossiping.
    \STATE $short\_peers$ is the set of this node's one-hop peers
    \STATE $table\_size$ is the minimum size of $short\_peers$
    \STATE $canidates$ is sorted  ascending by each node's distance to this node
    \STATE The closest member of $Canidates$ is popped and added to $short\_peers$
    \FORALL{$n$ in $Canidates$}
    	\STATE $c$ is the midpoint between this node and $n$
        \IF{Any node in $Peers$ is closer to $c$ than this node}
        	\STATE reject $n$ as a peer
        \ELSE
        	\STATE Add $n$ to $short\_peers$
        \ENDIF
    \ENDFOR
    \WHILE{$|candidates| >$ 0}
    	\FORALL{$candidates$}
    		\STATE add $candidates$ to the set of $long\_peers$
        	\IF{$|long\_peers| > table\_size^2$}
        		\STATE $long\_peers \leftarrow$ random subset of $long\_peers$ of size $table\_size^2$
        	\ENDIF
    	\ENDFOR
    \ENDWHILE
\end{algorithmic}
\end{algorithm}


Each cycle, nodes exchange their peerlists with a current neighbors and then recalculate their neighbors.  
A node combines the neighbor's peerlists and its own into a list of candidate neighbors.
This combined list is sorted from closest to furthest.
A new peerlist is then created starting with the first candidate from the list of candidates (which must allways be included).
The node then examines each of the remaining candidates in the sorted list and calculates the midpoint between the node and the candidate.
If any of the nodes in the new peerlist are closer to the midpoint than the canidate, the candidate is set aside.  Otherwise the candidate is added to the new peerlist.

This heuristic has the benefit of being fast and scalable into any metric space where a distance function and midpoint can be defined.
The distance metric used for this paper is the minimum distance in a multidimensional toroidal space.
Where $\vec{a}$ and $\vec{b}$ are locations in a $d$-dimensional toroidal space:
\[ Distance = \sqrt{\sum\limits_{i\in d} (\min(|\vec{a}_i-\vec{b}_i|, 1.0-|\vec{a}_i-\vec{b}_i|))^2}\]




This heuristic may fail.
Should a node be positioned between two other nodes such that the midpoint between those two nodes does not fall upon the shared face of thier voronoi regions then this heustic will not link the blocked peers. %picture
This is mitigated by our method of peer managment described in section D.


\subsection{Fast Vornoi Classifier}
VHash never actually calculates the polytopes that describe a node's vornoi region.
This is unnecessary and prohibitively expensive \cite{raynet}.
Rather VHash only ever assigns a given point to a vornoi region.
It does this by calculating the distance from that point to all canidate nodes.
The point falls into a node's voronoi region if it is the node to which it has the shortest distance.
Thus a node defines it's voronoi region by keeping a list of the peers that bound it.

\subsection{Peer Management}
VHash maintains two peer lists: Short Peers and Long Peers.
This is motivated by mitigating the error induced by the Distributed Greedy Voronoi Heuristic and providing robustness during churn\footnote{The disruption caused to the overlay by the continous joining, leaving, and failing of nodes.}.

Short Peers are the subset of the Delunay Peers generated by the Distributed Greedy Voronoi Heuristic. 
The number of short peers has no upper bound and in contrived cases, such as a single node surrounded by other nodes forming a hypersphere, it can grow quite high. Pevious work as found a usefull lower bound on peers to be $3d + 1$\cite{raynet}
Using a lower bound on the length of the short peer list corrects for errors in the approximation processes by including peers that otherwise would not be. 
An approximation of the number of nodes in the network is sufficient to generate an accurate expected number of short peers.  
This expected number of short peers is used as the lower bound for the size of short peer list.
Should the number of short peers generated by the Distributed Greedy Voronoi Heuristic be less than the lower bound for that set's size, the nearest peers not already included in the short peer list are added to the short peer list until it is of sufficient size.
Short peers are analogous to the predecessors/successors in other DHTs.

Long Peers correspond to the two hop peers of the node.
When a node learns about potential neighbors, but are not included in the short peer list, they may be included in the long peer list.  
The long peer list has a maximum size of the base length of the short peer list squared.  
For example, if the short peer list has a minimum size of 8, the long peer list has a maximum size of 64 entries.  
Long peers are not actively probed during maintenence and the cost of managing them is minimal.
 
We discuss how nodes learn about short and long peers in the Gossiping subsection.


\subsection{Maintenance via Gossiping}
Each node in the network performs maintenance periodically by `gossiping' with a randomly chosen neighbor.
When two nodes gossip with each other, they exchange their short peer lists with each other.
The node combines the lists of short peers\footnote{Naturally, nodes remove themselves and repitions from the candidates.} and uses the  Distributed Greedy Voronoi Heuristic to determine which of these candidates correspond to its neighbors along the Delaunay Triangulation.
The candidates determined not to be short peers become long peers.  
If resulting number of long peers exceeds the maximum size of the long peer list, a random subset of the maximum size is kept.

The formal algorithm for this process is described in \ref{gossip}.
This maintenance through gossip process is very similar to the gossip protocol used in \cite{raynet}.


\begin{algorithm}
\caption{Gossip Process}
\label{gossip}
\begin{algorithmic}[1]  % the number is how many 
	\STATE Node $n$ initiates the gossip.
	\STATE $neighbor \leftarrow$ random node from $n.short\_peers$
    \STATE $candidates \leftarrow n.short\_peers \cup neighbor.short\_peers$.
    \STATE $n$ and $neighbor$ each run Distributed Greedy Voronoi Heuristic using their own copy of $candidates$
\end{algorithmic} 
\end{algorithm}



\subsection{Handling Churn}
Churn is effects of the continous joining and exiting of nodes in the overlay.
DHTs must have some mechanisms to handle this process to maintain fault tolerance and routing.

Joining the network is a straightforward process.  
A prospective node must be able to communicate with at least one patron member of the DHT.  
The prospective node is assigned a location using the method described in Section III and uses the patron it knows to find the node responsible for it's assigned location, which we call the parent.
The prospective node sets the parent node as the lone member of the short peers and immediately gossips with the parent.
Subsequent gossips will refine the peer lists of the nodes affected by the join.

There is no ``polite'' exit from the network. VHash assumes nodes will fail and the distinction between an intended failure and unintended failure is unnecessary.
Should a node wish to crease participting in the network, it need only to crease sending and reciving messages.
Suppose a node $f$ fails or leaves the network; we assume it does so abruptly and without warning.
When one of $f$'s neighbors attempts to contact $f$ for gossiping or routing, failure to communicate with $f$ will prompt the neighbor to remove $f$ from its peers.  
The node the selects a different neighbor to gossip with or recomputes the peer closest to the location it was looking for. 
Should a short peer fail, routing around its failure is trivial due to knowledge about the long peers.

\subsection{Routing}
The proper forwarding peer for routing a message extends from the voronoi regions of the short peers and long peers.
Rather than attempt to know the true voronoi region of those peers we approximate the voronoi regions of those peers as if there where no other nodes in the network.
The resulting voronoi regions describe both the subset of the network for which that peer is responsible and the subset of the network for which it is the most efficent forwarding node.
The routing node determines the vornoi region into which the message's destination falls.
If this is itself, it handles the message accordingly; otherwise the routing node forwards the message to the responsible node.
This process is equivalent to a pre-computed and cached efficient routing algorithm (Algorrihm \ref{lookup}).
Even if our approximation for the Voronoi region of a peer is incorrect due to incomplete knowledge of the network, our approximation describes the most efficient forwarding path of a message  destined for that space availible.

\begin{algorithm}
\caption{Vhash Lookup}
\label{lookup}
\begin{algorithmic}[1] 
	\STATE $n$ is this node
	\STATE $m$ is a message addressed for $loc$
    \STATE $potential\_dests \leftarrow n \cup n.short\_peers$
    \STATE $c \leftarrow $ node in $ potential\_dests$ with shortest distance to $loc$
    \IF{$c$ == $n$}
    	\RETURN $n$
    \ELSE
        \RETURN c.lookup($loc$)
    \ENDIF
\end{algorithmic}
\end{algorithm}



\subsection{Algorithm Analysis}
The Distributed Greedy Voronoi Heuristic is very efficient in both terms of space and time.  Suppose a node is creating its short peer list from $k$ candidates.  The candidates must be sorted, which takes $O(k\cdot\lg(k))$ operations.  Then for each candidate, that the node must compute the midpoint between itself and the current candidate and then compare distances to that point between itself and all of the peers it has found so far.  This comes down to a cost of 

\[ k\cdot\lg(k) + k \text{ midpoint computations}  + k^{2} \text{ distance computations} \]


Since  $k$ is  bounded by $\Theta(\frac{\log N}{\log \log N} )$ \cite{bern1991expected} (the maximum number of peers stored neighbors allowed between two neighbors), we can translate the above to

\[O(\frac{\log^{2} N}{\log^{2} \log N} )\]

Because in the vast majority of cases, the number of peers is equal to the constant minium table size, which gives $k=O(1)$ in the expected case, the lower bound and expected complexities  are $\omega{}(1)$.

While previous work \cite{raynet} claims constant time approximation, the reality is that Raynet's leading constant is in the order of thousands as Monte-Carlo samples.  
While our algorithim has a greater asymtotic worst case cost, for all current realistic network sizes it will be more time efficent then raynet's approximation.



\section{Network Metric Embedding}

The VHash Space's flexible metric allows application of the network to solving routing issues.
Rather than be limited to the current ring-based metric, we intend to apply VHash to generating efficent routing and storage to a variety of applications.
Becuase information about a node can be encoded into a cordinate system we can encode an efficent routing scheme into the topological nowledge of the nodes in the network combined with the destination address.

By utilizing specific metric spaces we can design a generalized and fast model for a routing scheme in multiple sinarios.
For example, if VHash is used on a terrestrial Radio based network, recording latitude, logitude, altitude, and power levels allow distance between radios to be interpreted in terms of an approximation of signal strength rather than attempting to use eclidian distance as the distance metric.

Any metric space where inter-node distance and the midpoint between two nodes can be calculated efficently can utilize VHash's Distributed Greedy Voronoi Heuristic or ray-net's volume approximation heuristic should the preformance versus accuracy tradeoff be worthwhile.

VHash's interchangeable metric space is intended to allow applications of DHTs to areas of networking where DHT's preformance limitations previously prevented application.
Any routing scheme that can be described in a metric space where the ideal path on the network is approximated by a direct path in the metric space can be utilized by VHash.


Two types of information encoding are immediately apparent for use in VHashs: The usage of scalar values as indiviual axies and the embedding of a graph topology into an arbitrary number of dimensions.

As there has been no need for a distributed algorithim for graph embedding where each node represents a node in the graph we propose an inital naive approximation method based on Force-directed graph drawing \cite{Spring}.
The approximate distributed embedding algorithim utilizes VHash's maintence and join bevhavoirs.

As a node joins the VHash network for the first time it is assigned a random location in the toroidal metric space.
It joins the network and discovers an inital set of peers.
The node sends ping messages to each of it's peers to approximate latency.
The node calculates an ideal distance to each peer based on latency.
For each peer the node generates an error vector that would place the node at the idea distance.
These error vectors added to the node's current cordiantes and the node uses the result as a new location.
The node attempts to rejoin the network at this new location. Should the node's joining parent be it'self then the node changes the location it advertises to it's peers.
After a sufficent number of cycles nodes will find stable locations.
The approximation is an online algorithim and should be periodically re-applied to account for changes due to churn and congestion.


\begin{algorithm}
\caption{Decentralized Peer-to-Peer Spring Model}
\label{springalg}
\begin{algorithmic}[1] 
	\STATE $n$ is a node at location $loc$
    \STATE $total\_dist \leftarrow \sum distance(n,p),\forall p \in n.short\_peers$
    \STATE $total\_lat \leftarrow \sum latency(n,p),\forall p \in n.short\_peers$
    \STATE $unit\_lat \leftarrow total\_dist \div total\_lat$
    \FORALL{$p \in n.short\_peers$}
    	\STATE $ideal\_distance \leftarrow latency(n,p) \div unit\_lat$
        \STATE $error\_distance \leftarrow ideal_distance - dist(n,P)$
        \STATE $new\_location \leftarrow Location_n + error\_distance * Unit\_vector(n,p) $
    \ENDFOR
\end{algorithmic}
\end{algorithm}
Given node $N$, with random location $L_N$ and peers $P$
Claculate the sum of $N$'s distances to all nodes in $P$, $total_distance$
Claculate the sum of $N$'s latencies to all nodes in $P$, $total_latency$
Unit latency per distance is approximately $total_distance$/$total_latency$


For each node in $P$ calculate an ideal distance $Ideal_n$ = $latency_n / UnitLatency$
Calculate a location error vector of:
	For each node in peers, the vector that displaces N to the ideal distance.
Update $L_N$ as $L_N$ + $Error$
Rejoin the network at location $L_N$
Repeat Periodically.



%While encoding latitude and logitude as locations and using distance across the earth's surface as the distance metric may provide an approximation of a minmum latency overlay topology, this method cannot be used in more general network sinarios and may not describe the actual lowest latency path.

In the experimental section we preform an online distributed approximation of embedding the inter-node latency graph onto a two dimensional toroidal plane and use the resulting cordiantes in VHash to generate an approximate minimum latency overlay network. 


\section{Simulations}

We implemented two experimental tests to prove that VHash acts as as a stable DHT and is able to minimize the latency on lookup time.
While we have implemented functional software versions of the protocol, it was more economical to simulate VHash behavior.
Our simulation was of hit-rate over time as a VHash network was established from a random graph.  
This simulated a large network bootstrapping itself from a series of essentially random connections and into an ordered DHT.  
Our other simulation compared the distributions of latency in both the VHash and Chord DHTs to examine. 
We wanted to demonstrate the benefits from optimizing the network topology for minimum latency.

% Routing convergence tests
\subsection{Convergence Simulation}
Our first simulation demonstrates that VHash protocol converges to a stable overlay from an chaotic starting topology after a sufficient number of gossip cycles.  
We do this by showing that rate of successful lookups converges to 1.0.
We compare these results to RayNet \cite{raynet}, which proposed that a random k-connected graph would be a good demonstration of convergance for a DHT.

During the first two cycles of the simulation, each node bootstraps its short peer list by appending 10 randomly selected nodes. 
In all the cylces, the nodes perform a gossip with a random node from their short peers to recomute both their peer lists.
We then calculate the hitrate of successful lookups by simulating 2000 lookups from random nodes to random locations, as described in Alogirithm \ref{routesim}.
The hitrate at each cycle is $\frac{hits}{2000}$, where hits are the number of successful lookups.

\begin{algorithm}
\caption{Routing Simulation Sample}
\label{routesim}
\begin{algorithmic}[1]  % the number is how many 
	\STATE $start \leftarrow$ random node
	\STATE $dest \leftarrow$ random set of coordinates
    \STATE $ans \leftarrow$ node closest to $dest$
    \IF{$ans == start.lookup(dest)$}
    	\STATE increment $hits$
    \ENDIF
\end{algorithmic} 
\end{algorithm}


In order to show that leaving the size of the short peer list unbounded is not detrimental to the memory costs of nodes, we also kept track of the size of the short peer list at each cycle. 

%\subsection{Convergence Simulation Results}
We present our results in Figs ABCDEFG and Table \ref{tab:convtable}.
Our graphs show that VHash's overlay quickly contructs itself from a random configuration and that our hitrate approached 90\% by cycle $x$, regardless of dimension.
VHash consistently converged close to a hitrate of 1.0 by cycle $y$. 
In comparison, Raynet's routing converged to a perfect hitrate at around cycle 30 to 35 \cite{raynet} 
As the network size and number of dimensions increase, convergence slows, but not to a significant degree.




%pictures here


The average size of the short peer list is consistantly close to the set lower bound $3d+1$, regadless of the number of nodes in the network.   
The long peer tables typically are completely filled to capacity (and thus are not shown)


\begin{table}
\centering
\begin{tabular}{|r|r|r|r|}
\hline
Network Size & Dimensions & avg degree & max degree\\ \hline
500  & 2 & x & x \\ \hline
1000 & 2 & x & x \\ \hline
5000 & 2 & x & x \\ \hline
500  & 3 & x & x \\ \hline
1000 & 3 & x & x \\  \hline
5000 & 3 & x & x \\ \hline
500  & 4 & x & x \\ \hline
1000 & 4 & x & x \\  \hline
5000 & 4 & x & x \\ \hline
500  & 5 & x & x \\ \hline
1000 & 5 & x & x \\  \hline
5000 & 5 & x & x \\ \hline
\end{tabular}
\caption{Information about the size short peer lists, denoted degree here, at the last cycle of the convergence simulation.  For each pair of network size and dimension, we report the average degree in the network, as well as the largest.}
\label{tab:convtable}
\end{table}






% Hop distance test
\subsection{Latency Distribution Test}
The goal of our second set of experiments is to demostrate VHash's ability ot optimize a selected network metric: latency in this case.  
We compared VHash's performance to that of a more traditional DHT, Chord \cite{chord}.
Chord is a well established DHT with an $O(log(n)$ sized routing table and $O(log(n)$ lookup time measured in overlay hops.  
Rather than examine the number of hops on the overlay network as our primary metric like previous works, we are concerned with the actual latency lookups experience traveling through the \emph{underlay} network, the network the overlay is built upon.

For this experiment we constructed a 10000 node random scale free network (which has an approximate diameter of 3 hops) to act as an underlay \cite{cohen2000resilience} \cite{pastor2001epidemic} \cite{hagberg2004}.
We used a scale-free network as the underlay as it is a simplified model of the Internet's topology \cite{cohen2000resilience} \cite{pastor2001epidemic}.
From this underlay, we chose a random subset to be members of the overlay network.
We then measured the distance in underlay hops between 2000 random source-destination nodes from the overlay. 
VHash generates an embedding of the latency graph utilizing the distributed spring model algorithm described in Alogrithm \ref{springalg}, with the latency function is defined as the number of underlay hops.
%we should elaborate more as to why underlay hops are a good measure of latency


The overlay VHash has significantly shorter lookup latency than the Chord graph with smaller variation in latency.
Both are generally worse than the real network diameter of 3.
As the size of the network increases, the accuracy of the graph embedding falls (as scale free graphs difficult to embed.)
%pictures here







\subsection{Conclusions}
In general, in similar circumstances, VHash converges faster than RayNet.
Convergence rate varies with network size and number of dimensions of the VHash space.
Utilizing the current latency-embedding technique VHash is usefull for building DHT networks of reasonable size.

\section{Related Work}
%check table size consistancy 
While previous work has been done on applying Voronoi regions to DHTs and P2P applications, we have found no prior work on how to perform embedding of an inter-node latency graph.   

Backhaus et. al's  VAST \cite{Backhaus:2007:VAS:1326257.1326266} is a Voronoi-based P2P protocol designed for handling event messages in a massively multiplayer online videogame.  
Each node finds its neighbors by constructing a Voronoi diagram using Fortune's sweepline algorithm \cite{fortune1987sweepline}.  
VAST demonstrated that Voronoi diagrams could be used as the backbone  large-scale, although their work focused specifically on using 2-dimensional Voronoi diagrams.  
VHash approximates the Voronoi region rather than solving it, as higher dimension Voronoi regions are computationally expensive to solve.
In addition, VHash is designed to exploit network metrics.

The two DHT protocols developed by Beumont et al., VoroNet \cite{voronet} and RayNet \cite{raynet} are the closest comparisons to VHash.
VoroNet is based off Kleinberg's small world model \cite{kleinberg2000navigation} and achieves polylogarithmic lookup time.  
Each node in Voronet solves its Voronoi region to determine its neighbors and also maintains a link to a randomly chosen distant node.
Voronet focused specifically on the two-dimensional Vornoi computations and the techniques used would be too expensive in higher dimensions and were not resiliant to churn  \cite{raynet}.

RayNet \cite{raynet} was based on the work done on Voronet and is by far the most similar to VHash.  
Like VHash, RayNet does not solve for Voronoi regions, as that is prohibitively expensive.  
Unlike VHash, RayNet uses a Monte-Carlo method to approximate the volume of a node's Voronoi region.  
While effective and estimating the Voronoi region,  the volume-based Monte-Carlo approximation is expensive and requires multiple samples. 
RayNet does mention the idea of mapping attributes to each axis, but how this can be exploited is left as future work.



 

\section{Future Work}
\begin{itemize}
\item Explore applications of metric spaces.
\item Explore file managemet and backup method.
\item Explore chaching and replication stratagies.
\end{itemize}



\bibliographystyle{ieeetr}
\bibliography{P3DNS}
\end{document}
